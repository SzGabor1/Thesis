%Feladatkiiras
\noindent
\textsc{\textbf{Miskolci Egyetem}}\\
Gépészmérnöki és Informatikai Kar\\
Alkalmazott Matematikai Intézeti Tanszék\hspace*{4cm}\hfil \textbf{Szám:}

\vspace{0.5cm}
\begin{center}
\large\textsc{\textbf{Szakdolgozat Feladat}}
\end{center}
\vspace{0.5cm}

Szendrei Gábor (V9ZK10) BSc programtervező informatikus jelölt részére.

\bigskip
\noindent\textbf{A szakdolgozat tárgyköre:} Játékprogramozás

\bigskip
\noindent\textbf{A szakdolgozat címe:} Kétdimenziós játék fejlesztése Pygame keretrendszerrel

\bigskip
\noindent\textbf{A feladat részletezése:}

\medskip

\emph{A szakdolgozat célja egy Pygame-en alapuló kétdimenziós akció- és kalandjáték megtervezése és fejlesztése, a grafikától az implementációig. A dolgozat bemutatja az implementációhoz használt technológiákat, különös tekintettel a Python nyelvre és Pygame könyvtárra, miért alkalmasak kalandjáték fejlesztésére, és összehasonlítja őket egyéb alternatívákkal.
A dolgozat bemutatja az akció- és kalandjátékok általános jellemzőit, és ezt vesszük inspirációnak a fejlesztett játék megtervezéséhez. Mint kalandjátékokra jellemző, a fejlesztett játékban a cselekménysort egy küldetés rendszer adja, ehhez megvalósításra kerülnek különböző interakciók NPC-kkel. Az alapvető játékfunkciókon (karakter irányítása, pályával való interakció, támadás) felül a még jobb játékélmény érdekében megvalósításra kerülnek egyéb játékfunkciók is, mint a pálya animációja és hangeffektek, illetve egy pillanatállj funkció. Továbbá megvalósításra kerül egy bejelentkezési rendszer, mely lehetőséget ad az előrehaladás mentésére, illetve a legjobb egyéni eredmények mentésére egy online adatbázisban és versenyre más játékosok rekordjaival.
}

%\medskip

%\emph{(Kisebb tagolás lehet benne, hogy jól nézzen ki.)}

\vfill

\noindent\textbf{Témavezető:} Dr. Vadon Viktória, adjunktus

% \noindent\textbf{Konzulens(ek):} (akkor kötelezõ, ha a témavezetõ nem valamelyik matematikai tanszékrõl való; de persze lehet egyébként is)\newline

\bigskip
\noindent\textbf{A feladat kiadásának ideje:} 2023. szeptember 21.

%\noindent\textbf{A feladat beadásának határideje:}

\vspace{1.5cm}

\hfill\makebox[6cm]{\dotfill}

\hfill\makebox[6cm]{szakfelelős}

\clearpage

\vspace*{1cm}  
\begin{center}
\large\textsc{\textbf{Eredetiségi Nyilatkozat}}
\end{center}
\vspace*{2cm}  

Alulírott \textbf{Szendrei Gábor}; Neptun-kód: \texttt{V9ZK10} a Miskolci Egyetem Gépészmérnöki és Informatikai Karának végzős Programtervező informatikus szakos hallgatója ezennel büntetőjogi és fegyelmi felelősségem tudatában nyilatkozom és aláírásommal igazolom, hogy \textit{Szakdolgozat Címe}
című szakdolgozatom saját, önálló munkám; az abban hivatkozott szakirodalom
felhasználása a forráskezelés szabályai szerint történt.

\medskip
Tudomásul veszem, hogy szakdolgozat esetén plágiumnak számít:
\begin{itemize}
\item szószerinti idézet közlése idézőjel és hivatkozás megjelölése nélkül;
\item tartalmi idézet hivatkozás megjelölése nélkül;
\item más publikált gondolatainak saját gondolatként való feltüntetése.
\end{itemize}

Alulírott kijelentem, hogy a plágium fogalmát megismertem, és tudomásul veszem, hogy
plágium esetén szakdolgozatom visszautasításra kerül.

\vspace*{3cm}

\noindent Miskolc, \makebox[2cm]{\dotfill}. év \makebox[2cm]{\dotfill}. hó \makebox[2cm]{\dotfill}. nap

\vspace*{3cm}

\hfill\makebox[6cm]{\dotfill}

\hfill\makebox[6cm]{Hallgató}



\clearpage

\newcommand{\ki}{témavezető(k)}
\newsavebox{\alairas}
\begin{lrbox}{\alairas}
\begin{tabular}{c@{\hspace{2cm}}c}
\makebox[4cm]{\dotfill} & \makebox[5cm]{\dotfill} \\
dátum & \ki \\
\end{tabular}
\end{lrbox}
\newcommand{\dotline}{\makebox[5cm]{\dotfill}}
\newcommand{\shortdotline}{\makebox[3.5cm]{\dotfill}}

\noindent 1.
\begin{tabular}[t]{cl}
\multirow{2}{*}{A szakdolgozat feladat módosítása}
&szükséges (módosítás külön lapon) \\
& nem szükséges\\[1ex]
\end{tabular}

\begin{center}
\usebox{\alairas}
\end{center}

\smallskip

\noindent 2. A feladat kidolgozását ellenőriztem:

\begin{center}
\begin{tabular}{c@{\hspace*{2cm}}c}
témavezető (dátum, aláírás): & konzulens (dátum, aláírás):\\
\dotline & \dotline \\
\dotline & \dotline \\
\dotline & \dotline 
\end{tabular}
\end{center}

\smallskip

\noindent 3. A szakdolgozat beadható:

\begin{center}
\usebox{\alairas}
\end{center}

\noindent 4.
\begin{tabular}[t]{@{}l@{\hspace*{1mm}}l@{\hspace*{1mm}}l}
A szakdolgozat & \shortdotline & szövegoldalt\\
              & \shortdotline & program protokollt (listát, felhasználói leírást)\\
              & \shortdotline & elektronikus adathordozót (részletezve)\\
              & \shortdotline \\
              & \shortdotline & egyéb mellékletet (részletezve)\\
              & \shortdotline 
\end{tabular}
\newline tartalmaz.

\begin{center}
\usebox{\alairas}
\end{center}

\noindent 5.
\begin{tabular}[t]{ll}
\multirow{2}{*}{A szakdolgozat bírálatra} & bocsátható\\
& nem bocsátható\\
\end{tabular}

\smallskip

\noindent A bíráló neve: \makebox[8cm]{\dotfill}

\renewcommand{\ki}{szakfelelős}
\begin{center}
\begin{tabular}{c@{\hspace{2cm}}c}
\makebox[4cm]{\dotfill} & \makebox[5cm]{\dotfill} \\
dátum & \ki \\
\end{tabular}
\end{center}

\noindent 6.
\begin{tabular}[t]{lll}
A szakdolgozat osztályzata \\
& a témavezető javaslata: & \makebox[2.5cm]{\dotfill} \\
& a bíráló javaslata: & \makebox[2.5cm]{\dotfill} \\
& a szakdolgozat végleges eredménye: & \makebox[2.5cm]{\dotfill}
\end{tabular}

\bigskip\bigskip

\noindent Miskolc, \makebox[4cm]{\dotfill} \hfill \makebox[8cm]{\dotfill} 

\hfill \makebox[8cm]{a Záróvizsga Bizottság Elnöke} 
