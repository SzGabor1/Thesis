\chapter{Bevezetés}

\indent \indent A számítógépes játékok már régóta a szórakozás egyik legnépszerűbb formáját képviselik. Az idők során folyamatos fejlődésen mentek keresztül, és ma már számtalan lehetőséget kínálnak a játékosoknak.

A számítógépes játékok lehetővé teszik számunkra, hogy elmerüljünk olyan világokban, amelyek különböző fantáziavilágokban játszódnak, izgalmas cselekmények részesei lehetünk, és akár versenyezhetünk a világ minden tájáról játszós játékosokkal. Az interaktivitás a játékok egyik fő jellemzője, hiszen a játékosok döntéseket hozhatnak, irányíthatják a karaktereket és befolyásolhatják a játék alakulását, így személyes élményeket és kalandokat élhetnek át.

A számítógépes játékokban a grafika és a hang is kulcsfontosságú szerepet játszik. A modern grafikus motorok és a kiváló minőségű hangeffektek valósághű és lenyűgöző világokat teremtenek, amelyekbe a játékosok könnyedén belefeledkezhetnek. A 3D-s grafika és a virtuális valóság (VR) technológia pedig még inkább fokozza a valóságérzetet, lehetővé téve, hogy teljes mértékben elmerüljünk a játék világában.

A számítógépes játékoknak számos műfaja létezik, így mindenki megtalálhatja a saját ízlésének megfelelő játékot. Akciójátékok, stratégiai játékok, szerepjátékok, sportjátékok, horror játékok és még sok más közül választhatunk, mindegyiknek saját kihívásai és élményei vannak.

Emellett a számítógépes játékoknak társadalmi szerepük is van. A többjátékos módban lehetőség van csapatban játszani, vagy akár versenyezni más játékosokkal online. Ez segít kapcsolatokat építeni és barátokat szerezni.

Az eSport egyre népszerűbbé válik, és hatalmas növekedési potenciállal rendelkezik. Versenyek és ligák alakulnak ki, ahol a legjobb játékosok hatalmas pénzdíjakért versenyeznek, és mindezt élvezhetik a nézők is.

Összességében a számítógépes játékok kiváló lehetőséget nyújtanak a szórakozásra, és a technológia folyamatos fejlődésének köszönhetően még évekig fontos szerepet fognak játszani a szórakoztatóiparban.

A szakdolgozatom célja egy 2-dimenziós akció-kaland játék fejlesztése, melyben a játékosok teljesítsenek különféle izgalmas küldetéseket, miközben felfedezik a szigetet és barlangjait és kihívásokkal teli kalandokban vesznek részt.

Dolgozatomban részletes betekintést nyújtok a Python programozási nyelv és a pygame könyvtár használatába. Emellett összehasonlító elemzést végezek a pygame és a Ren'py könyvtárak között, és bemutatom a játék implementációjának részleteit is.