\chapter{Bevezetés}

A számítógépes játékok már hosszú évek óta népszerű szórakozási formaként vonult be a köztudatba,, valamint az idők alkalmával nagy fejlődésen mentek keresztül. Ezek a virtuális világok lehetőséget kínálnak a játékosoknak arra, hogy elmerüljenek különféle fantázia világokban, kihívásokkal, továbbá kalandokkal teli történetekben, esetleg éppen versenyezzenek más játékosokkal a világ az összes tájáról.

Az egyik kulcsfontosságú vonása a számítógépes játékoknak az interaktivitás. A játékosok nem csupán a történet megfigyelői, hanem részesei is a cselekménynek. Képesek döntéseket hozni, irányítani a karakterüket, valamint befolyásolni a játékmenetet. Ezáltal a játékok lehetővé teszik a perszonális élményeket, ezen kívül kalandokat.

A grafika, ezen kívül a hang is domináns szerepet játszik a számítógépes játékokban. A fejlett grafikus motorok, továbbá a nagyszerű kvalitású hanghatások lehetővé teszik, hogy a játékosok valósághű, valamint lenyűgöző világokban kalandozzanak. A 3D-s grafika, ezen kívül a VR (virtuális valóság) technológia tovább növeli a valóságérzetet, továbbá lehetővé teszi, hogy a játékosok teljes mértékben elmerüljenek a játék világában.

A számítógépes játékok sok féle műfajban elérhetők, így mindenki megtalálhatja a saját ízlésének megfelelőt. Vannak akció játékok, stratégiai játékok, szerepjátékok, sportjátékok, horror játékok, továbbá még sok más. Mindegyik műfaj saját kihívásokat, valamint élményeket kínál.

Ezzel együtt a számítógépes játékoknak közösségi aspektusa is van. Sok játék lehetőséget nyújt a többjátékos módra, ahol játékosok csapatokat alkothatnak, esetleg egymás ellen versenyezhetnek. Az online játékok lehetővé teszik, hogy egyének világszerte kapcsolódjanak egymáshoz, barátságokat kössenek, valamint versenyeket vívjanak.

Azonban lényeges megjegyezni, hogy a számítógépes játékoknak is vannak kritikusai. Néhányan aggodalmukat fejezik ki a játékfüggőség, a túlzott képernyőidő, továbbá a játékok esetleg előforduló negatív hatásai okán. Emiatt releváns, hogy a játékokat mértékkel játsszuk, ezen kívül figyeljünk a kiegyensúlyozott életmódra.

Összességében a számítógépes játékok releváns szórakozási forma, mely lehetővé teszi a játékosok részére, hogy elmerüljenek különféle világokban, kihívásokkal szembesüljenek, ezen kívül kapcsolatokat alakítsanak ki a játékos közösségekben. A technológia permanens fejlődése továbbra is izgalmas újabb lehetőségeket kínál a játékosoknak, valamint így a számítógépes játékoknak még hosszú ideig lényeges szerepük lesz a szórakoztatóiparban.

A szakdolgozatom célja egy 2-dimenziós akció-kaland játék fejlesztése, melyben a játékosok teljesítsenek különféle izgalmas küldetéseket, miközben felfedezik a szigetet és barlangjait és kihívásokkal teli kalandokban vesznek részt.

Dolgozatomban részletes betekintést nyújtok a Python programozási nyelv és a pygame könyvtár használatába. Emellett összehasonlító elemzést végezek a pygame és a Ren'py könyvtárak között, és bemutatom a játék implementációjának részleteit is.