\chapter{Összefoglalás}

\indent \indent A szakdolgozatom célja egy 2 dimenziós akció-kaland játék fejlesztése volt a pygame könyvtár használatával.

A dolgozat elkészítése során részletesen megismerkedtem a python programozási nyelvvel, a pygame játékfejlesztő könyvtárral, illetve a modern és gyors FastAPI web frameworkkel amely segítségével egy autentikációs rendszert, online játékállapot-mentést is készítettem a játékhoz. Nagyon jó döntésnek tartom, hogy a játék fejlesztéséhez az előbb felsorolt eszközöket használhattam, és biztos vagyok benne, hogy újra ezeket fogom választani a jövőben is, ha hasonló projektbe vágom a fejszémet.

Saját és közeli ismerőseim véleménye alapján, akik kipróbálták a játékot, szerintük is nagyon jól sikerült. Az úgynevezett indie típúsú játékok kategóriájába tökéletesen beleillik. Sikerült a mai igényeknek megfelelő letisztult és modern kinézet kialakítása. A játék jelenlegi állapota lehetővé teszi a felhasználónak, hogy felhasználói fiókot hozzon létre, internetes, illetve internet nélküli használatot egyaránt. Továbbá változatos küldetések teljesítését, szörnyekkel és más ellenséges karakterekkel való küzdelmet.

Az alkalmazás további kihívásokat és izgalmas feladatokat tartogat a jövőre nézve. A játék továbbfejlesztésével szeretném a játékélményt még jobbá tenni, és új funkciókat hozzáadni a játékhoz. Ilyen funkció lehet a táska rendszer továbbfejlesztése, a pályagenerálás, illetve az ellenségek képességeinek egyedi megvalósítása, de fontos megemlíteni a grafika megtervezését is, amelynek legalább ugyanannyi a szerepe a felhasználói élményre nézve, mint a különböző játékmechanikáknak.  

Abban egészen biztos vagyok, hogy a szakdolgozat elkészítése során szerzett rengeteg hasznos ismeretet, tapasztalatot alkalmazni tudom majd a jövőben is.

\chapter{Summary}

\indent \indent The aim of my thesis was to develop a 2-dimensional action-adventure game using the pygame library.

While working on the thesis, I gained detailed knowledge of the Python programming language, the pygame game development library, as well as the modern and fast FastAPI web framework, through which I created a login system and online game state saving feature for the game. I consider it a very good decision to use the tools mentioned above for game development, and I am confident that I will choose them again in the future if I embark on a similar project.

Based on the opinions of myself and close acquaintances who have tried the game, they also believe that it turned out great. It fits perfectly into the category of so-called indie games. I managed to achieve a sleek and modern design that meets today's standards. The current state of the game allows users to create user accounts for both online and offline use. Additionally, it involves completing various missions and engaging in battles with monsters and other enemies.

The game holds further challenges and exciting tasks for the future. With further development of the game, I aim to enhance the gaming experience and add new features. Such features could include further development of the inventory system, area generation, unique implementation of enemy abilities, and it is important to mention the design of graphics, which plays just as important a role in user experience as various game mechanics.

I am confident that the valuable knowledge and experience gained during the making of this project will be applicable in the future.