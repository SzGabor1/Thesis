\chapter{Összefoglalás}

A szakdolgozatom célja egy 2 dimenziós akció-kaland játék fejlesztése volt a pygame könyvtár használatával.

A dolgozat elkészítése során részletesen megismerkedtem a python programozási nyelvvel, a pygame játékfejlesztő könyvtárral, illetve a modern és gyors FastAPI web frameworkkel amely segítségével egy bejelnetkzezési rendszert, online játékállapotmentést is készítettem a játékhoz. Nagyon jó döntésnek tartom, hogy a játék fejlesztéséhez az előbb felsorolt eszközöket használhattam, és biztos vagyok benne, hogy újra ezeket fogom választani a jövőben is, ha hasonló projektbe vágom a fejszémet.

Saját és közeli ismerőseim véleményenye alapján akik kipróbálták a játékot szerintük is nagyon jól sikerült. Az úgynevezett indie típúsú játékok kategóriájába tökéletesen beleillik. Sikerült a mai igényeknek megfelelően letisztult és modern kinézet kialakítás megvalósítása. A játék jelenlegi állapota lehetővé teszi a felhasználónak, hogy felhasználói fiókot hozzon létre, internet illetve internet nélküli használatot egyaránt. Továbbá változatos küldetések teljesítését, szörnyekkel és más ellenséges karakterekkel való küzdelmet.

Az alkalmazás további kihívásokat és izgalmas feladatokat tartogat a jövőre nézve. A játék továbbfejlesztésével szeretném a játékélményt még jobbá tenni, és új funkciókat hozzáadni a játékhoz. Ilyen funkció lehet a táska rendszer továbbfejlesztése, a pálya generálás, illetve az ellenségek képességeinek egyedi megvalósítása, de fontos megemlíteni a grafika megtervezését is, amelynek legalább ugyanannyi szerepe a felhasználói élményre nézve van mint a különböző játék mechanikáknak.  

Abban egészen biztos vagyok, hogy a szakdolgozat elkészítése során szerzett rengeteg hasznos ismeretet, tapasztalatot alkalmazni tudom majd a jövőben is.